У Вяземского была квартира окнами на Тверской бульвар. Пушкин
очень любил ходить к нему в гости. Придет "--- и сразу прыг на
подоконник, свесится из окна и смотрит. Чай ему тоже  туда,  на
окно, подавали. Иной раз там и заночует. Ему даже матрац купили
специальный,  только  он  его не признавал. "<\textit{К чему}, "--- говорит,"---
\textit{такие роскоши?}">. И спихнет матрац с подоконника.  А  потом  всю
ночь вертится, спать не дает.

\begin{center}
    * * *
\end{center}

    Гоголь  переоделся  Пушкиным,  пришел к Пушкину и позвонил.
Пушкин открыл ему и кричит: "<\textit{Смотри, Арина Родионовна,  я  пришел!}">.

\begin{center}
    * * *
\end{center}
    Лермонтов хотел у Пушкина жену увести. На Кавказ. Всё Смотрел на
нее из"~за колонн, смотрел\dots Вдруг устыдился своих желаний.
"<\textit{Пушкин},  "---  думает,  "---  \textit{зеркало русской революции, а я "---
свинья}">. Пошел, встал перед ним на колени и  говорит:  "<\textit{Пушкин,
где твой кинжал? Вот грудь моя}">. Пушкин очень смеялся.

\begin{center}
    * * *
\end{center}

    Однажды Пушкин стрелялся с Гоголем. Пушкин говорит:

    "--* Стреляй первым ты.

    "--* Как я? Нет, ты.

    "--* Ах, я! Нет, ты!

    Так и не стали стреляться.

\begin{center}
    * * *
\end{center}

    Лев Толстой очень любил детей. Однажды он шел по  Тверскому
бульвару  и  увидел впереди Пушкина. "<\textit{Конечно, это уже не ребенок,
это уже подросток}, "--- подумал Лев Толстой, "--- \textit{все равно, дай
догоню и поглажу по головке}">. И побежал догонять Пушкина.  Пушкин
же, не зная толстовских намерений, бросился наутек. Пробегая
мимо городового, сей страж порядка был возмущен неприличной
быстротою бега в людном месте и бегом устремился вслед с  целью
остановить.  Западная пресса потом писала, что в России литераторы
подвергаются преследованиям со стороны властей.

\begin{center}
    * * *
\end{center}

    Однажды Лермонтов купил яблок, пришел на Тверской бульвар и
стал угощать присутствующих дам. Все брали и говорили  "<мерси">.
Когда же подошла Наталья Николаевна с сестрой Александриной, от
волненья  он так задрожал, что яблоко упало к ее ногам (Натальи
Николаевны, а не Александрины). Одна из собак схватила яблоко и
бросилась бежать. Александрина, конечно, побежала за  ней.  Они
были одни "--- впервые в жизни (Лермонтов, конечно, а не Александрина
с собачкой). Кстати, она (Александрина) ее не догнала.

\begin{center}
    * * *
\end{center}

    Однажды  Пушкин  решил  испугать  Тургенева  и спрятался на
Тверском бульваре под лавкой. А Гоголь тоже решил в  этот  день
испугать  Тургенева, переоделся Пушкиным и спрятался под другой
лавкой. Тут Тургенев идет. Как они оба выскочат!\dots

\begin{center}
    * * *
\end{center}

    Лев Толстой очень любил детей. Однажды он играл с ними весь
день и проголодался. "<\textit{Сонечка}, "--- говорит, "--- \textit{а, ангелочек,  сделай
мне тюрьку}">. Она возражает: "<\textit{Левушка, ты же видишь, я "<Войну
и  мир"> переписываю}">. "<\textit{А"~а"~а}, "--- возопил он, "--- \textit{так я и знал,
что тебе мой литературный фимиам дороже моего  "<Я">}.  И  костыль
задрожал в его судорожной руке.

\begin{center}
    * * *
\end{center}

    Однажды Пушкин написал письмо Рабиндранату Тагору. "<Дорогой
далекий друг, "--- писал он, "--- я Вас не знаю, и Вы меня не знаете.
Очень хотелось бы познакомиться. Всего хорошего. Саша>".

    Когда письмо принесли, Тагор предавался самосозерцанию. Так
погрузился, хоть режь его. Жена толкала, толкала, письмо подсовывала  
"--- не видит. Он, правда, по"~русски читать не умел. Так и
не познакомились.

\begin{center}
    * * *
\end{center}

    Однажды Федору Михайловичу Достоевскому, царствие  ему  небесное,  
исполнилось  150  лет.  Он очень обрадовался и устроил
день рождения. Пришли к нему все писатели, только почему"~то все
наголо обритые. У одного Гоголя усы нарисованы. Ну хорошо,  выпили,
закусили, поздравили новорожденного, царствие ему небесное,
сели играть в вист. Сдал Лев Толстой "--- у каждого  по  пять
тузов. Что за черт? Так не бывает. "<\textit{Сдай"~ка, брат Пушкин, лучше
ты}">.  "<\textit{Я},  "--- говорит, "--- \textit{пожалуйста, сдам}">. И сдал. У каждого по
шесть тузов и по две пиковые дамы. Ну и  дела\dots  "<\textit{Сдай"~ка  ты,
брат Гоголь}">. Гоголь сдал\dots Ну, знаете\dots Даже и нехорошо сказать\dots 
Как"~то получилось так\dots Нет, право, лучше не надо.

\begin{center}
    * * *
\end{center}

    Однажды  Федор  Михайлович Достоевский, царствие ему небесное, 
сидел у окна и курил. Докурил и выбросил окурок  из  окна.
Под  окном  у  него была керосиновая лавка. И окурок угодил как
раз в бидон с керосином. Пламя, конечно, столбом. В  одну  ночь
пол"~Петербурга сгорело. Ну, посадили его, конечно. Отсидел, вышел,
идет  в  первый же день по Петербургу, навстречу "--- Петрашевский.
Ничего ему не сказал, только пожал руку и в глаза посмотрел. Со значением.

\begin{center}
    * * *
\end{center}

    Снится однажды Герцену сон. Будто иммигрировал он в  Лондон
и живется ему там очень хорошо. Купил он, будто, собаку бульдожей
породы. И до того злющий пес "--- сил нет. Кого увидит, на того  
бросается.  И  уж  если догонит, вцепится мертвой хваткой?
все, можешь бежать заказывать панихиду. И вдруг, будто  он  уже
не  в  Лондоне,  а в Москве. Идет по Тверскому бульвару, чудище
свое на поводке держит, а навстречу Лев Толстой. И надо же, тут
на самом интересном месте пришли декабристы и разбудили.

\begin{center}
    * * *
\end{center}

    Гоголь только под конец жизни о душе задумался, а смолоду у
него вовсе совести не было. Однажды невесту в карты проиграл  и
не отдал.

\begin{flushright}
    <\dots>
\end{flushright}