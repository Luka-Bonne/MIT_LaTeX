МОЛОДОЙ  ЧЕЛОВЕК, УДИВИВШИЙ СТОРОЖА \\
    "--* Ишь ты, "--- сказал сторож,  рассматривая 
муху. "--* Ведь если ее помазать столярным кле"-
ем, то ей, пожалуй, и конец придет. Вот ведь
история! От простого клея! \\
    "--* Эй ты, леший! "--- окликнул сторожа моло-
дой человек в желтых перчатках.
    Сторож сразу же понял, что это обращают"-
ся к нему, но продолжал смотреть на
муху. \\
    "--* Не  тебе что  ли  говорят?  "---  крикнул
опять молодой человек. "--* Скотина! \\
    Сторож раздавил муху пальцем и, не пово"-
рачивая головы к молодому человеку, сказал:
    "--* А ты чего, срамник, орешь"~то?  Я и так
слышу. Нечего орать-то! \\
    Молодой человек почистил перчатками свои
брюки и деликатным голосом спросил:
    "--* Скажите,  дедушка,  как  тут пройти на
небо? \\
    Сторож  посмотрел  на молодого человека,
прищурил один глаз,  потом  прищурил другой,
потом почесал  себе бородку, еще раз посмот"-
рел на молодого человека и сказал: \\
    "--* Ну, нечего тут задерживаться, проходи"-
те мимо. \\
    "--* Извините, "---  сказал молодой человек, "--*
ведь я по срочному делу. Там для меня  уже и
комната приготовлена. \\
    "--* Ладно,  *---  сказал сторож, *--* покажи би"-
лет. \\
    "--* Билет не у меня; они говорили, что ме"-
ня и так пропустят, "--- сказал  молодой  чело"-
век, заглядывая в лицо сторожу. \\
    "--* Ишь ты! "--- сказал сторож. \\
    "--* Так как же? "--- спросил молодой человек.
"--* Пропустите? \\
    "--* Ладно, ладно, "--- сказал сторож. "--*  Иди"-
те. \\
    "--* А как пройти-то? Куда? "--- спросил моло"-
дой человек. "--* Ведь я и дороги"~то не знаю. \\
    "--* Вам куда нужно? "--- спросил сторож,  де"-
лая строгое лицо. \\
    Молодой  человек  прикрыл  рот ладонью и
очень тихо сказал: \\
    "--* На небо! \\
    Сторож наклонился вперед, подвинул  пра"-
вую ногу, чтобы  встать потверже, пристально
посмотрел  на  молодого  человека  и  сурово
спросил: \\
    "--* Ты чего? Ваньку валяешь? \\
    Молодой человек улыбнулся, поднял руку в
желтой  перчатке,  помахал  ею над головой и
вдруг исчез. \\
    Сторож понюхал воздух.  В  воздухе пахло
жжеными перьями.
    "--* Ишь  ты!  "---  сказал  сторож, распахнул
куртку, почесал себе живот, плюнул в то мес"-
то,  где стоял молодой  человек,  и медленно
пошел в свою сторожку.